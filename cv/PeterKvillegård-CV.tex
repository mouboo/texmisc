\documentclass[11pt,a4paper]{article}
\usepackage[utf8]{inputenc}
\usepackage[top=0.5in, bottom=1in, left=1in, right=1in]{geometry}
\usepackage{url}
\usepackage{graphicx}
\usepackage{parskip}
\usepackage[usenames,dvipsnames]{xcolor}
\usepackage{titlesec}
\usepackage{ebgaramond}
\usepackage{tabularx}
\usepackage{enumitem}
\usepackage{multirow}
\usepackage{fix-cm}
\usepackage{array}
\usepackage{tikz}

%\usepackage{charter}
%\usepackage{times}

\makeatletter
\newcommand\CVsize{\@setfontsize\Huge{60}{60}}
\makeatother  

% Link formatting
\usepackage{hyperref}
\definecolor{linkcolour}{rgb}{0,0.2,0.6}
\hypersetup{colorlinks,breaklinks,urlcolor=linkcolour, linkcolor=linkcolour}

%Section formatting
\titleformat{\section}{\Large\scshape\raggedright}{}{0em}{}[\titlerule]
\titlespacing{\section}{0pt}{3pt}{3pt}

%\newcommand{\sbt}{\,\begin{picture}(-1,1)(-1,-3)\circle*{3}\end{picture}\ }

\begin{document}
\pagestyle{empty}

%--------------------TITLE-------------

\hspace{-7pt}\begin{tikzpicture}
\node[inner sep=0pt] (l) at (0,0) {};
\node[inner sep=0pt] (m) at (0.5\linewidth,0) {};
\node[inner sep=0pt] (r) at (1.008\linewidth,0) {};

\node[inner sep=0pt, anchor=west] at (l) {\CVsize CV};

\node at (m) {\parbox{\linewidth}{\centering \Huge Peter Kvillegård \\ \LARGE Fysioterapeut\\[5pt] \normalsize Tel: 0760330019 \hspace{0.3em} \textbar \hspace{0.3em} E-post: peterkvillegard@gmail.com}};

\node[inner sep=0pt, anchor=east] (pic) at (r)
{\includegraphics[width=.16\linewidth]{portrait}};
\end{tikzpicture}

%--------------------SECTIONS-----------------------------------
%Section: Utbildning
\vspace{-5pt}
\section{Utbildning}
\hspace{-4pt}
\begin{tabularx}{\textwidth}{r|X}	
	\multirow{2}{*}{
		\begin{tabular}{@{}r@{}}
			Okt 2017\\
			-- Juni 2020\\
			\footnotesize{(Pågående)}\\
			\footnotesize{}
		\end{tabular}
	}
	& Kandidatexamen i Fysioterapi, Lunds universitet, Lund\\
	&\footnotesize{\vspace{-5pt}
		 \begin{itemize}[leftmargin=10pt, topsep=-12.5pt]

		\item Praktikplats termin 4: Geriatrikavdelning 31, Malmö. Praktikperioden innehöll mycket funktionsanalyser, förflyttningsträning, och dokumentation i Melior. 
		
		\item Praktikplats termin 4: Barn- och ungdomshabiliteringen, Lund. Praktikperioden innehöll bl.a. utprovning av hjälpmedel, CPUP-utvärdering, och träningsprogram.
		
		\item Praktikplats termin 5: Medicinavdelning 2, Malmö.
		Fokus på interprofessionellt arbete, eget ansvar, bedömning av funktion, och journalföring.
		
		\item Praktikplats termin 6: Kärråkra vårdcentral, Eslöv. Självständigt arbete med bedömning, behandling och uppföljning av patienter i primärvården.
		
		\item Praktikplats termin 6: Papegojelyckan HSL SBÄ, Lund.
		Fysisk aktivitet med äldre, fallprevention, utvärdering av behov av hälpmedel.
				
		\end{itemize}\vspace{-30pt} 
	}\\
	\multicolumn{2}{c}{} \\
\end{tabularx}

%Section: Övriga utbildningar
\section{Övriga utbildningar}
\begin{tabularx}{\textwidth}{r|X}
	
	Okt. 2016& Certifierad massageterapeut, Axelsons gymnastiska institut, Göteborg\\
	-- Mars 2017& \footnotesize{Utbildningen innehöll bl.a. klassisk svensk massage, taktil massage, och barnmassage.} \\
	\multicolumn{2}{c}{} \\
	
	Aug. 2005& Kurser på Ölands folkhögskola, Skogsby \\
	--Maj 2006&\footnotesize{\vspace{-5pt}
		\begin{itemize}[leftmargin=10pt, topsep=-12.5pt]
			\item Hälsa och livskvalitet
			\item Skrivarskolan
		\end{itemize}\vspace{-30pt} 
	}\\
	\multicolumn{2}{c}{} \\
	
	Sep. 2004& Kurser på Malmö högskola, Malmö\\
	-- Dec. 2004&\footnotesize{\vspace{-5pt}
		\begin{itemize}[leftmargin=10pt, topsep=-12.5pt]
			\item Hälsa och samhälle, 7.5 hp
			\item Medicinsk vetenskap, 7.5 hp
		\end{itemize}\vspace{-30pt} 
	}\\
	\multicolumn{2}{c}{} \\	
		
	Jan. 2003& Kurs på Linnéuniversitetet, Kalmar\\
	-- Mars 2003&\footnotesize{\vspace{-5pt}
		\begin{itemize}[leftmargin=10pt, topsep=-12.5pt]
			\item English for science and technology, 7.5 hp
		\end{itemize}\vspace{-30pt} 
	}\\
	
	\multicolumn{2}{c}{} \\
	
	Aug. 1998& Gymnasium, Naturvetenskapsprogrammet teknisk gren, Mönsterås gymnasium,\\
	-- Juni 2001&Mönsterås\\
	\multicolumn{2}{c}{} \\
\end{tabularx}

%Section: Yrkeslivserfarenhet
\section{Yrkeslivserfarenhet}
\begin{tabularx}{\textwidth}{r|X}	
	Mars 2017 & Massageterapeut, Högsby \\
	-- Sept 2017 &\footnotesize{Som massageterapeut satte jag ihop behandlingar baserade på kundens behov och kroppsliga förutsättningar. }\\
	\multicolumn{2}{c}{} \\

	Feb 2007 & Maskinoperatör/programmerare på Stranda AB, Södra Bäckebo \\
	-- Sept 2016&\footnotesize{Jag producerade precisionsdetaljer till olika industrier genom maskinstyrd skärande bearbetning av metall. Jag hade speciellt ansvar för mätinstrument och dokumentering av processer, vilket krävde god samarbetsförmåga och kommunikation med kollegor.}\\
	\multicolumn{2}{c}{} \\
	
	Okt 2001& Massaläggare på Scania AB, Oskarshamn \\
	-- Mars 2004&\footnotesize{Arbetet innehöll stora krav på noggrannhet och händighet.}\\
	\multicolumn{2}{c}{} \\

\end{tabularx}

%Section: Övriga meriter
\section{Övriga meriter}
\begin{tabularx}{\textwidth}{X}
	\vspace{-7pt}
\begin{itemize}[leftmargin=0.8em]
	\item Har dokumenterat i Melior, PMO, och Procapita.
	\item Erfarenhet av olika typer av yoga (vinyasa, acro, yin, bikram), samt aerobisk och muskelstärkande träning.
	\item Kandidatuppsats: ``EMG-analys av m. deltoideus och m. infraspinatus vid utåt\-rotations\-övningar med och utan adduktionskraft.''
	\item Har gemonfört Socialstyrelsens webbutbildning om Förskrivning av hjälpmedel.
	\item Har genomfört Svenskt demenscentrums webbutbildning Demens ABC.
	\item Har körkort.
\end{itemize}
\end{tabularx}

%Section: Styrkor
\section{Styrkor}
\begin{tabularx}{\textwidth}{X}
	\vspace{-7pt}
	\begin{itemize}[leftmargin=0.8em]
		\item Bra på att bemöta människor
		\item Noggrann och tålmodig
		\item Stort intresse för att ta till mig ny kunskap
		\item Mycket god förmåga att tala och skriva på engelska
	\end{itemize}
\end{tabularx}

\section{Intressen och aktiviteter}
\begin{tabularx}{\textwidth}{X}
Gå i skogen, läsa alla typer av böcker, programmering, besöka British Columbia med min flickvän, vegetarisk matlagning, spela gitarr, och ta hand om djur.\\
\\
\end{tabularx}


\section{Referenser}
\begin{tabularx}{\textwidth}{X}
Erika Rönningberg, handledare på praktikplatsen Kärråkra vårdcentral, termin 6 \\ Tel: 0702800983, E-post: erika.ronningberg@skane.se \\
\\
Anna Pyrih, handledare på praktikplatsen Papegohlyckan HSL SBÄ, termin 6\\ Tel: 046-3596182, E-post: anna.pyrih@lund.se \\
\end{tabularx}

\end{document}