\documentclass[10pt,a4paper]{article}
\usepackage[utf8]{inputenc}
\usepackage[top=0.7in,bottom=1in,left=1in,right=1in]{geometry}
\title{Kan vi lita på våra kliniska test vid axelsmärta?}
\author{}
\date{}

\begin{document}
\pagestyle{empty}	

\noindent Peter Kvillegård, 2020-03-01. Antal ord: 577.
\vspace{-1em}
{\let\newpage\relax\maketitle}
\vspace{-3em}

\noindent Apeldoorn et al.$^{1}$ har undersökt om olika terapeuter får samma resultat med samma patienter när de utför vanligt förekommande kliniska axeltest. De fann att av 21 test så uppfyllde bara Full Can Test kraven på både bedömar\-sam\-stämmighet och reliabilitet.

\subsubsection*{Bakgrund}
När terapeuter undersöker patienter som söker för axelsmärta så finns många olika kliniska test till hjälp. Med dessa verktyg vill man ta reda på vilka strukturer som är påverkade och mekanismerna bakom smärtan, så att rätt behandling och lämpliga hemövningar kan väljas. 

I alla dessa test finns ett moment av subjektivt tolkande, man frågar sig exempelvis ``var det här ett smärtsvar?'' eller ``är det här ledspelet normalt?'' Att tolka patientens reaktioner kan vara svårt, och man kan misstänka att olika terapeuter bedömer samma situation på olika sätt.

\subsubsection*{Syfte}
Studiens författare ville ta reda på om vanliga kliniska axeltest var tillräckligt tillförlitliga genom att låta olika terapeuter undersöka samma patienter, och sedan analysera hur ofta terapeuterna var överens eller inte.

\subsubsection*{Metod}
Man testade 113 patienter i primärvården som sökte för axelsmärta, men som inte hade rhizopati, neurologiska orsaker, rheumatism, demens, och/eller allvarliga sjukdomar såsom maligniteter.

36 terapeuter delades in i par, som sedan parvis, men oberoende av varandra, undersökte patienter med samtliga 21 axeltest. För varje test kunde man sedan sortera resultaten från terapeutparen och räkna hur många gånger terapeuterna höll med varandra om att testet var positivt, hur många gånger de höll med varandra om att testet var negativt, och hur många gånger de inte höll med varandra.

Det enklaste sättet att mäta samstämmighet är att räkna hur ofta två terapeuter håller med varandra. Det kan dock vara missvisande om resultaten är skevt fördelade mellan positiva och negativa resultat, och därför räknade man istället ut specifik bedömarsamstämmighet, vilket innebär hur ofta \emph{en terapeut får ett visst resultat, givet att den andra terapeuten också fått det resultatet.} Man räknade också ut Cohens $\kappa$, vilket är ett mått på reliabilitet som kontrollerar för att samstämmighet kan uppstå slumpmässigt.

\subsubsection*{Resultat}
Tillfredställande positiv specifik bedömarsam\-stämmighet, dvs att terapeuterna var överens om att testet hade positivt resultat, fanns för sju av axeltesten: Empty Can Test, Full Can Test, Active Compression Test, Neer Test, mSAT, Relocation Test, och Release Test.

Motsvarande för negativ specifik bedömarsamstämmighet, dvs att båda terapeuterna fick negativt resultat, fanns för tio test: External Rotation Resistance Test, Full Can Test, Kim Test, Biceps Load II Test, Internal Rotation Resistance Strength Test, Load and Shift Test, Acromioclavicular Joint Stress Test, Sulcus Sign Test, CRT, och Glenohumeral Internal Rotation Deficit Test. 

Bara ett test fick tillfredställande resultat för alla mått, inklusive reliabilitet: Full Can Test.

\subsubsection*{Diskussion} 
Man försäkrade sig om att resultaten inte påverkades av terapeuternas kunskap om testen, eller att testerna i sig påverkade patienterna att uppvisa olika symtom för olika terapeuter.

\subsubsection*{Reflektion}
Det är alltså inte så att våra kära test är värdelösa, men resultaten visar att vi måste vara ödmjuka och öppna för att sanningen inte är så lättfångad. Från denna studie kan man ta med sig att kliniska test är en viktig del av undersökningen och till stor hjälp, men att enbart kliniska test för sig själva inte är en garanti för att upptäcka eller avfärda patologiska tillstånd i axeln. För att komma fram till rätt diagnos, behandling och rehabilitering måste vi pussla ihop allt från anamnes och lednadsvanor till funktionsanalys och kliniska test för att få en helhetsbild av patientens situation och möjligheter. Detta är en stor men nödvändig utmaning för oss fysioterapeuter.

\subsubsection*{Referens}
\begin{enumerate}
\item Apeldoorn AT, Den Arend MC, Schuitemaker R, Egmond D, Hekman K, Van Der Ploeg T, et al. Interrater agreement and reliability of clinical tests for assessment of patients with shoulder pain in primary care. Physiother Theory Pract. 2019 Mar 22:1-20.
\end{enumerate}
\end{document}