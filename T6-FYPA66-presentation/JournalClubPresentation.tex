\documentclass[aspectratio=169,12pt,handout,usenames,dvipsnames]{beamer}
\usepackage[utf8]{inputenc}
\usepackage[swedish]{babel}
\usepackage{amsmath}
\usepackage{amsfonts}
\usepackage{amssymb}
\usepackage{graphicx}
\usetheme{default}
%\usefonttheme{serif}
\setbeamertemplate{navigation symbols}{}
\usepackage{color,soul}

\usepackage{xcolor}
\usepackage{graphicx}
\usepackage{multirow}
\usepackage{array}
\usepackage{booktabs}
\usepackage{colortbl}

\newcommand{\allpagesbool}{1}


%Introduktion

%Syfte/hypotes

%Material och metoder

%Resultat

%Diskussion

%Presentationen ska avslutas med:

%Kvalitetsgranskning. Välj granskningsmall (checklista) som du tycker passar den artikel du valt Motivera din bedömning av studiens kvalitet. Ange källa/källor.

%Klinisk relevans/applicerbarhet. Motivera din bedömning av studiens kliniska relevans/applicerbarhet. 

\begin{document}
	%\author{}
	\title{Interrater agreement and reliability of clinical tests\\ for assessment of patients with shoulder pain\\ in primary care.}
	\subtitle{Apeldoorn AT, Den arend MC, Schuitemaker R, et al. \\
			  Physiother Theory Pract. 2019;:1-20.\\[2em]
			  \textcolor{black}{(Interbedömarsamstämmighet och reliabilitet av kliniska test\\ 
			  för bedömning av patienter med skuldersmärta i primärvården)}}
	%\logo{}
	%\institute{}
	\date{}
	%\subject{}
	%\setbeamercovered{transparent}
	%\setbeamertemplate{navigation symbols}{}
	\frame[plain]{\vspace{2em}\maketitle} 

\begin{frame}
	\frametitle{Introduktion}
	\begin{itemize}
		\item Många patienter söker för axelsmärta i primärvården
		\item Undersökning med kliniska test (ex. Neers, Hawkins, Sulcus sign, etc)
		\item Oklart hur reliabla dessa test är
	\end{itemize}
\end{frame}

\begin{frame}
	\frametitle{Syfte}
	\begin{itemize}
		\item Ta reda på om vanligt förekommande kliniska axeltest är tillförlitliga\\
		genom att räkna ut specifik interbedömarsamstämmighet och reliabilitet
	\end{itemize}
\end{frame}

\begin{frame}
	\frametitle{(Vad är ``specifik interbedömarsamstämmighet''?)}
	{
		\setlength\extrarowheight{2pt}
		\noindent\begin{tabular}{c|ccl}
			& Terapeut 1  & Terapeut 2 & \\
			\cline{1-3}
			Patient 1 & pos & pos & \textcolor{white}{$\leftarrow$ Positiv samstämmighet}\\
			Patient 2 & neg & neg &\textcolor{white}{$\leftarrow$ Negativ samstämmighet}\\
			Patient 3 & pos & neg &\textcolor{white}{$\leftarrow$ Inte samstämmiga}\\
			Patient 4 & neg & pos &\textcolor{white}{$\leftarrow$ Inte samstämmiga}\\
		\end{tabular}
	}\par
	\vspace{2em}
	\par \textcolor{white}{Specifik interbedömarsamstämmighet = om en terapeut får ett resultat, hur ofta får den andra terapeuten samma resultat?}
	\vspace{1em}
	\par \textcolor{white}{I denna studie: $>$0.75 räknas som tillräckligt bra.}
\end{frame}

\begin{frame}
	\frametitle{(Vad är ``specifik interbedömarsamstämmighet''?)}
	{
		\setlength\extrarowheight{2pt}
		\noindent\begin{tabular}{c|ccl}
			& Terapeut 1  & Terapeut 2 & \\
			\cline{1-3}
			Patient 1 & pos & pos & \textcolor{black}{$\leftarrow$ Positiv samstämmighet}\\
			Patient 2 & neg & neg &\textcolor{white}{$\leftarrow$ Negativ samstämmighet}\\
			Patient 3 & pos & neg &\textcolor{white}{$\leftarrow$ Inte samstämmiga}\\
			Patient 4 & neg & pos &\textcolor{white}{$\leftarrow$ Inte samstämmiga}\\
		\end{tabular}
	}\par
	\vspace{2em}
	\par \textcolor{white}{Specifik interbedömarsamstämmighet = om en terapeut får ett resultat, hur ofta får den andra terapeuten samma resultat?}
	\vspace{1em}
	\par \textcolor{white}{I denna studie: $>$0.75 räknas som tillräckligt bra.}
\end{frame}

\begin{frame}
	\frametitle{(Vad är ``specifik interbedömarsamstämmighet''?)}
	{
		\setlength\extrarowheight{2pt}
		\noindent\begin{tabular}{c|ccl}
			& Terapeut 1  & Terapeut 2 & \\
			\cline{1-3}
			Patient 1 & pos & pos & \textcolor{black}{$\leftarrow$ Positiv samstämmighet}\\
			Patient 2 & neg & neg &\textcolor{black}{$\leftarrow$ Negativ samstämmighet}\\
			Patient 3 & pos & neg &\textcolor{white}{$\leftarrow$ Inte samstämmiga}\\
			Patient 4 & neg & pos &\textcolor{white}{$\leftarrow$ Inte samstämmiga}\\
		\end{tabular}
	}\par
	\vspace{2em}
	\par \textcolor{white}{Specifik interbedömarsamstämmighet = om en terapeut får ett resultat, hur ofta får den andra terapeuten samma resultat?}
	\vspace{1em}
	\par \textcolor{white}{I denna studie: $>$0.75 räknas som tillräckligt bra.}
\end{frame}

\begin{frame}
	\frametitle{(Vad är ``specifik interbedömarsamstämmighet''?)}
	{
		\setlength\extrarowheight{2pt}
		\noindent\begin{tabular}{c|ccl}
			& Terapeut 1  & Terapeut 2 & \\
			\cline{1-3}
			Patient 1 & pos & pos & \textcolor{black}{$\leftarrow$ Positiv samstämmighet}\\
			Patient 2 & neg & neg &\textcolor{black}{$\leftarrow$ Negativ samstämmighet}\\
			Patient 3 & pos & neg &\textcolor{black}{$\leftarrow$ Inte samstämmiga}\\
			Patient 4 & neg & pos &\textcolor{black}{$\leftarrow$ Inte samstämmiga}\\
		\end{tabular}
	}\par
	\vspace{2em}
	\par \textcolor{white}{Specifik interbedömarsamstämmighet = om en terapeut får ett resultat, hur ofta får den andra terapeuten samma resultat?}
	\vspace{1em}
	\par \textcolor{white}{I denna studie: $>$0.75 räknas som tillräckligt bra.}
\end{frame}

\begin{frame}
	\frametitle{(Vad är ``specifik interbedömarsamstämmighet''?)}
	{
		\setlength\extrarowheight{2pt}
		\noindent\begin{tabular}{c|ccl}
			& Terapeut 1  & Terapeut 2 & \\
			\cline{1-3}
			Patient 1 & pos & pos & \textcolor{black}{$\leftarrow$ Positiv samstämmighet}\\
			Patient 2 & neg & neg &\textcolor{black}{$\leftarrow$ Negativ samstämmighet}\\
			Patient 3 & pos & neg &\textcolor{black}{$\leftarrow$ Inte samstämmiga}\\
			Patient 4 & neg & pos &\textcolor{black}{$\leftarrow$ Inte samstämmiga}\\
		\end{tabular}
	}\par
	\vspace{2em}
	\par \textcolor{black}{Specifik interbedömarsamstämmighet = om en terapeut får ett resultat, hur ofta får den andra terapeuten samma resultat?}
	\vspace{1em}
	\par \textcolor{black}{I denna studie: $>$0.75 räknas som tillräckligt bra.}
\end{frame}

\begin{frame}
	\frametitle{(Vad är ``reliabilitet''?)}
	\begin{itemize}
		\item Cohen's $\kappa$: ett mått på reliabilitet som tar hänsyn till att samstämmighet kan bero på slumpen
		\item I denna studie:
		\begin{itemize}
			\item $<$0.2: Poor
			\item 0.21--0.40: Fair
			\item 0.41--0.60: Moderate
			\item 0.61--0.80: Good
			\item 0.81--1.00: Very good 	
		\end{itemize}
	\end{itemize}
\end{frame}

\begin{frame}
	\frametitle{Metod}
	\begin{itemize}
		\item 36 terapeuter indelade i par
		\item 113 patienter med axelsmärta
		\item 21 kliniska test
	\end{itemize}
\end{frame}

\begin{frame}
	\frametitle{Metod / Kliniska test}
	\begin{columns}
		\begin{column}{0.5\textwidth}
			\begin{itemize}
				\item Scapula Position
				\item External Rotation Resistance Test
				\item Empty Can Test (Jobe test)
				\item Full Can Test
				\item Active Compression Test (O’Brien’s test)
				\item Neer Test
				\item Hawkins-Kennedy Test
				\item Kim Test
				\item Biceps Load II Test
				\item Internal Rotation Resistance Strength Test (Zaslav Test)
			\end{itemize}
		\end{column}
		\begin{column}{0.5\textwidth}
			\begin{itemize}
				\item Load and Shift Test
				\item Acromioclavicular Joint Stress Test
				\item Modified Scapular Assistance Test
				\item Scapular Retraction Test
				\item Impingement Relief Test
				\item Sulcus Sign Test
				\item Apprehension Test
				\item Relocation Test
				\item Release Test
				\item Combined Reduction Test
				\item Glenohumeral Internal Rotation Deficit Test
			\end{itemize}
		\end{column}
	\end{columns}
\end{frame}

\begin{frame}
		\frametitle{Resultat}
	
	{
		\scriptsize
		\setlength\extrarowheight{0pt}
		\noindent\begin{tabular}{lccc}
			\multirow{2}{0.15\linewidth}{\\Test}&
			\multirow{2}{0.15\linewidth}{\centering Positiv specifik\\samstämmighet}&
			\multirow{2}{0.15\linewidth}{\centering Negativ specifik\\samstämmighet}& 
			\multirow{2}{0.15\linewidth}{\centering \\Cohens $\kappa$}\\
			&&&\\
			%Test & Positiv specifik samstämmighet & Negativ specifik samstämmighet & Cohens K  \\
			\hline
			Scapula position & 0.71 & 0.73 & 0.44\rule{0pt}{2.6ex}\\
			External rotation resistance test & 0.72 & \cellcolor{green}{0.78} & 0.50\\
			Empty can test (Jobe test) & \cellcolor{green}{0.79} & 0.72 & 0.51\\
			Full can test & \cellcolor{green}{0.75} & \cellcolor{green}{0.87} & \cellcolor{green}{0.62}\\
			Active compression test (O'Briens test) & \cellcolor{green}{0.79} & 0.67 & 0.46\\
			Neer test & \cellcolor{green}{0.83} & 0.59 & 0.43\\
			Hawkins-Kennedy test & 0.74 & 0.59 & 0.33\\
			Kim test & 0.56 & \cellcolor{green}{0.78} & 0.34\\
			Biceps load II test & 0.40 & \cellcolor{green}{0.90} & 0.31 \\
			Internal rotation resistance strength test & 0.64 & \cellcolor{green}{0.83} & 0.47 \\		
			Load and shift test & 0.48 & \cellcolor{green}{0.91} & 0.40\\
			Acromioclavicular joint stress test & 0.64 & \cellcolor{green}{0.82} & 0.47\\
			Modified scapular assistance test & \cellcolor{green}{0.77} & 0.62 & 0.39\\
			Scapular retraction test & 0.56 & 0.68 & 0.25\\
			Impingement relief test & 0.62 & 0.72 & 0.35\\
			Sulcus sign test & 0.45 & \cellcolor{green}{0.91} & 0.36 \\
			Apprehension test & 0.70 & 0.62 & 0.32\\
			Relocation test & \cellcolor{green}{0.76} & 0.50 & 0.27\\
			Release test & \cellcolor{green}{0.81} & 0.64 & 0.46\\
			Combined reduction test & 0.48 & \cellcolor{green}{0.78} & 0.26\\
			Glenohumeral internal rotation deficit test & 0.68 & \cellcolor{green}{0.86} & 0.54\\
			\bottomrule
			
		\end{tabular}
	}
\end{frame}

\begin{frame}
	\frametitle{Diskussion}
	\begin{itemize}
		\item Resultaten liknar andra studier som gjorts
		\item Undviker felkällor, bl.a. genom symptomatiskt stabila patienter
	\end{itemize}
\end{frame}

\begin{frame}
	\frametitle{Kvalitetsgranskning}
	Uppfyller samtliga punkter i Guidelines for Reporting Reliability and Agreement Studies (GRRAS).

\end{frame}

\begin{frame}
	\frametitle{Klinisk relevans}
	\begin{itemize}
		\item Bra att känna till begränsningar när man tolkar resultat
		\item Se kliniska test som \emph{en del} av den samlade bilden
	\end{itemize}
\end{frame}

\end{document}