\documentclass[10pt,a4paper]{letter}
\usepackage[utf8]{inputenc}
\usepackage[swedish]{babel}
\usepackage{lipsum}
\usepackage[left=1.5in, right=1.5in, top=0in, bottom=1.5in]{geometry}
%\usepackage{newtxtext}
\usepackage{tgtermes}
%\date{30 mars 2020}



\address{Peter Kvillegård \\ Ällingavägen 18 \\ 227 34 Lund \\ Tel: 0760330019 \\ peterkvillegard@gmail.com} 
\signature{Peter Kvillegård} 


\begin{document} 
	\begin{letter}{} 
		
\opening{\textbf{Ansökan om arbete som fysioterapeut inom distriktsrehabilitering i Kalmar.}}
%\vspace{1em}
Hej!

Jag skriver angående positionen som fysioterapeut som jag såg annonserad på Platsbanken. Jag blev glad av att se precis det jag letat efter!

Just nu håller jag på att avsluta mina fysioterapeutstudier på Lunds universitet och förväntat \mbox{datum} för examen är 2020-06-12. Jag är en hängiven student och jag har ofta fördjupat mig mer än vad som krävdes inom ämnen som anatomi, neurologi, och fysiologi. Detta lärande\-intresse har jag haft mycket nytta av i kliniska resonemang när jag tagit hand om patienter.

Under mina praktikplatser på bl.a. Kärråkra vårdcentral i Eslöv, Medicinavdelning 2 och Geriatrisk avdelning 31 i Malmö fick jag värdeful erfarenhet som fick mig att känna att rehabilitering verkligen är vad jag vill syssla med. Där lärde jag mig de olika aspekterna av det dagliga arbetet som fysioterapeut i praktiken. Jag tyckte om att applicera min kunskap och bli mer bekant med rådande kliniska riktlinjer och behandlingsmetoder. Mina handledare har sagt att jag har en bra patientkontakt och empatiskt förhållningssätt, och att jag är bra på att resonera kring patientfall. Jag tror att min omtanke till patienterna också visar sig i hur uppmärksam och genuint intresserad jag är att grundligt undersöka och behandla besvär. Jag tänker mig att min noggrannhet och vilja att lära mig mer skulle vara en tillgång i er verksamhet.

Jag tycker egentligen att allt inom fysioterapi är intressant, men kanske främst det omväxlande och utmanande arbetet med rehabilitering i alla dess former.

I min tidigare arbetslivserfarenhet lärde jag mig organisatoriska färdigheter, såsom vikten av god kommunikation, samarbete, och att hantera ansvar. Detta har format mig att vara en pålitlig och skötsam anställd.

I framtiden ser jag fram emot att fortsätta lära mig mer och skulle gärna gå kurser för att bättre kunna hjälpa patienter.

Jag ser fram emot att höra från er för att prata mer om vad jag kan tillföra er arbetsplats!
	
\vspace{1em}
\hspace{0.1\linewidth}Med vänliga hälsningar,

\hspace{0.1\linewidth}Peter Kvillegård
		 
 
\end{letter} 
\end{document}