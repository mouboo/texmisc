\documentclass[10pt,a4paper]{letter}
\usepackage[utf8]{inputenc}
\usepackage[swedish]{babel}
\usepackage{lipsum}
\usepackage[left=1.5in, right=1.5in, top=0in, bottom=1.5in]{geometry}
%\usepackage{newtxtext}
\usepackage{tgtermes}
%\date{30 mars 2020}



\address{Peter Kvillegård \\ Ällingavägen 18 \\ 227 34 Lund \\ Tel: 0760330019 \\ peterkvillegard@gmail.com} 
\signature{Peter Kvillegård} 


\begin{document} 
	\begin{letter}{Referens: A945015} 
		
\opening{\textbf{Ansökan om arbete som fysioterapeut med placering på barnavdelning i Helsingborg.}}
%\vspace{1em}
Hej!

Jag skriver angående positionen som fysioterapeut som jag såg annonserad på Platsbanken. Jag tänkte direkt att detta skulle passa mig väl!

Just nu håller jag på att avsluta mina fysioterapeutstudier på Lunds universitet och förväntat \mbox{datum} för examen är 2020-06-12. Under utbildningen fick vi vid ett tillfälle möjlighet att välja område och då valde jag att söka mig till verksamheter med barn och ungdomar, för jag tycker det är spännande och roligt att jobba med barn. Jag är en hängiven student och jag har ofta fördjupat mig mer än vad som krävdes inom ämnen som anatomi, neurologi, och fysiologi, inte minst om andning och lungornas funktion. Detta lärande\-intresse har jag haft mycket nytta av i kliniska resonemang om patienter.

Under min praktikplats på Barn- och ungdomshabiliteringen i Lund fick jag värdefull erfarenhet om hur det är att bemöta barn, och att se deras behov. Där lärde jag mig de olika aspekterna av det dagliga arbetet som fysioterapeut i praktiken. Jag tyckte om att applicera min kunskap och bli mer bekant med rådande kliniska riktlinjer och behandlingsmetoder. Mina handledare har sagt att jag har en bra patientkontakt och ett empatiskt förhållningssätt. Jag tror att min omtanke till patienter också visar sig i hur uppmärksam och genuint intresserad jag är att grundligt undersöka och behandla besvär. Jag tänker mig att min noggrannhet och vilja att lära mig är tillgångar i er verksamhet.

I min tidigare arbetslivserfarenhet lärde jag mig organisatoriska färdigheter, såsom vikten av god kommunikation, samarbete, och att hantera ansvar. Detta har format mig att vara en pålitlig och skötsam anställd.

I framtiden ser jag fram emot att fortsätta lära mig mer och skulle gärna gå kurser för att bättre kunna hjälpa patienter.

Jag ser fram emot att höra från er för att prata mer om vad jag kan tillföra er arbetsplats!
	
\vspace{1em}
\hspace{0.1\linewidth}Med vänliga hälsningar,

\hspace{0.1\linewidth}Peter Kvillegård
		 
 
\end{letter} 
\end{document}