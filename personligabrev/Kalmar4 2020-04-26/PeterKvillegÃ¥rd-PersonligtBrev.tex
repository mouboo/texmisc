\documentclass[10pt,a4paper]{letter}
\usepackage[utf8]{inputenc}
\usepackage[swedish]{babel}
\usepackage{lipsum}
\usepackage[left=1.5in, right=1.5in, top=0in, bottom=1.5in]{geometry}
%\usepackage{newtxtext}
\usepackage{tgtermes}
%\date{30 mars 2020}



\address{Peter Kvillegård \\ Ällingavägen 18 \\ 227 34 Lund \\ Tel: 0760330019 \\ peterkvillegard@gmail.com} 
\signature{Peter Kvillegård} 


\begin{document} 
	\begin{letter}{Referens: 2020/91} 
		
\opening{\textbf{Ansökan om arbete som fysioterapeut på länssjukhuset i Kalmar}}
%\vspace{1em}
Hej!

Jag skriver angående positionen som fysioterapeut som jag såg annonserad på Platsbanken. Jag blev glad när jag läste beskrivningen för det är precis vad jag letar efter!

Just nu håller jag på att avsluta mina fysioterapeutstudier på Lunds universitet och förväntat \mbox{datum} för examen är 2020-06-12. Jag är en hängiven student och har ofta fördjupat mig långt mer än vad som krävdes i olika ämnen. Detta lärande\-intresse har jag haft mycket nytta av i praktiken, exempelvis vid kliniska resonemang om patienter.

Under mina praktikplatser på Kärråkra vårdcentral i Eslöv, Medicinavdelning 2 och Geriatrisk avdelning 31 i Malmö fick jag värdeful erfarenhet som bekräftade mitt intresse för rehabilitering och för att lära mig om patienters behov. Där lärde jag mig de olika aspekterna av det dagliga arbetet som fysioterapeut i praktiken. Jag tyckte om att applicera min kunskap och bli mer bekant med rådande kliniska riktlinjer och behandlingsmetoder. Jag är egentligen intresserad av de flesta inriktningar inom fysioterapi, men jag har märkt att jag tyckt mest om ``detektivarbetet'' med att förstå hur olika kroppsliga tillstånd påverkar patienters liv och att se dem bli bättre genom rätt behandling.

Mina handledare har sagt att jag har en bra patientkontakt och empatiskt förhållningssätt. Jag tror att min omtanke till patienterna också visar sig i hur uppmärksam och genuint intresserad jag är att grundligt undersöka och behandla besvär. Jag tänker mig att min noggrannhet och vilja att lära mig är en bra grund för att utföra ett gott arbete.

I min tidigare arbetslivserfarenhet lärde jag mig organisatoriska färdigheter, såsom vikten av god kommunikation, samarbete, och att hantera ansvar. Detta har format mig att vara en pålitlig och skötsam anställd. 

I framtiden ser jag fram emot att fortsätta lära mig mer och skulle gärna gå kurser för att bättre kunna hjälpa patienter.

Jag ser fram emot att höra från er för att prata mer om vad jag kan tillföra er arbetsplats!
	
\vspace{1em}
\hspace{0.1\linewidth}Med vänliga hälsningar,

\hspace{0.1\linewidth}Peter Kvillegård
		 
 
\end{letter} 
\end{document}